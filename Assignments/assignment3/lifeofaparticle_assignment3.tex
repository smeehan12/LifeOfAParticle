\newpage
\textbf{Modelling the Particle In a Box} 
\newline
We discussed at great length the solution of the Schrodinger equation for the case of a particle in a one-dimensional box with sides at $x=$0 and $x=$a (Go review \href{http://www.fisica.net/mecanica-quantica/Griffiths\%20-\%20Introduction\%20to\%20quantum\%20mechanics.pdf}{Griffith's QM - Chapter 2.2} if you are not sure of how the calculation proceeds.).  This is the model by which we want to describe a particle, namely the answer to the question ``Where is the particle?''.  This will be done using the wave function $\Psi(x,t)$ of the particle, which itself may be complex valued as is composed as a linear superposition of the eigenfunctions 
\begin{displaymath}
 \Psi(x,t)=\displaystyle\sum_{n=0}^{\infty} c_{n}  \phi_{n}(t)  \psi_{n}(x) = \displaystyle\sum_{n=0}^{\infty} c_{n}  e^{-\frac{i}{\hbar}E_{n}t} sin(\frac{n\pi}{a}x)
\end{displaymath}
However, the probability rule, which will be a function describing the PDF $P(x)$ of the location of the particle, must be a real valued function from which a single event (a single collapse of the wave function) can be viewed as the generation of a single number from this distribution $P(x)$.  Therefore, we must translate this complex function into a real-valued function.  Two reasonable ways to do this are as follows :
\begin{itemize}
\item Square Then Sum : Take the square modulus of each of the eigenfunctions and then add these squared terms.
   \begin{itemize}
     \item $P_{A}(x,t)=\displaystyle\sum_{n=0}^{\infty}  c_{n}  \phi_{n}(t)  \psi_{n}(x) c_{n}^{*}  \phi_{n}^{*}(t)  \psi_{n}^{*}(x) $
   \end{itemize}
\item Sum Then Square : Add the individual eigenfunctions and then take the square modulus of this sum.
   \begin{itemize}
     \item $P_{B}(x,t)=\displaystyle\sum_{n=0}^{\infty}  c_{n}  \phi_{n}(t)  \psi_{n}(x) \times \displaystyle\sum_{m=0}^{\infty}  c_{m}^{*}  \phi_{m}^{*}(t)  \psi_{m}^{*}(x)$
   \end{itemize}
\end{itemize}

If you are feeling overwhelmed looking at these equations,its alright, we are going to deal with a simplified system.  Let's imagine that we know that the particle is initially placed in the box in a state $\Psi(x,t)$ which is only composed of the $E_1$ and $E_2$ eigenstates.  So the wave function is simply 
\begin{displaymath}
\Psi(x,t)=c_{1}\phi_{1}(t)\psi_{1}(x)+c_{2}\phi_{2}(t)\psi_{2}(x)
\end{displaymath}
First, describe how these two different possibly probability rules differ (if they do) qualitatively?  Is there time dependence to one of the probability rules?  What if you set $t=$0, meaning that the observation is made immediately after you put the particle in the box?  Does the time dependence go away?
\newline
\newline
For this simplified system, you have been provided with a set of 5000 data measurements (included on the assignment page).  Your goal is to determine which one of these two probabily rule transformations $P_{A}$ or $P_{B}$ is the one that really occurs in nature.  To this end, you should probably start by examining the data, either by using descriptive statistics, or maybe making a histogram.  Can you observe anything about the data just from this?  Now try to generate a predictive set of data according to the two different models that you have for the probability, assuming that the measurements being made are performed immediately at $t=$0.  To fully describe the model, there are therefore two separate questions to answer
\begin{itemize}[noitemsep]
\item What is the probability rule that comes from nature?
\item What are the coefficients $c_1$ and $c_2$? (To make things less involved, pretend that we know that $c_1$ and $c_2$ are positive and between [$0,1$].)
\end{itemize}
To go about this, we will need a way to compare your ensemble of predictions to that of the observations.  This can be done by first casting the obervations or predictions in the form of two histograms ($p$ and $o$) and then performing a calculation of the $\chi^{2}$ of these two histograms where
\begin{displaymath}
\chi^{2}=\displaystyle\sum_{i \in bins(p,o)} \frac{(p_{i}-o_{i})^2}{\sigma_{p_{i}}^{2}+\sigma_{o_{i}}^{2}}
\end{displaymath}
and ($p_{i}$, $o_{i}$) is the bin content of $p$ and $o$ at bin $i$ and($\sigma_{p_{i}}$, $\sigma_{o_{i}}$) are the corresponding errors on these bins.