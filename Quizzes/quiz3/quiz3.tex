
\documentclass[12pt]{article}
%	options include 12pt or 11pt or 10pt
%	classes include article, report, book, letter, thesis

\usepackage[margin=0.5in]{geometry}

\setlength{\parindent}{0pt}

\usepackage{hyperref}

%for writing of code in blocks like
%\begin{lstlisting}
%   .......
%\end{lstlisting}
\usepackage{listings}
\usepackage{color}
\usepackage{enumitem}

\definecolor{dkgreen}{rgb}{0,0.6,0}
\definecolor{gray}{rgb}{0.5,0.5,0.5}
\definecolor{mauve}{rgb}{0.58,0,0.82}

\lstset{frame=tb,
  language=C++,
  aboveskip=3mm,
  belowskip=3mm,
  showstringspaces=false,
  columns=flexible,
  basicstyle={\small\ttfamily},
  numbers=none,
  numberstyle=\tiny\color{gray},
  keywordstyle=\color{blue},
  commentstyle=\color{dkgreen},
  stringstyle=\color{mauve},
  breaklines=true,
  breakatwhitespace=true,
  tabsize=3
}
%%%%%%%%%%%%%%%%%%%%%%

\title{Life of a Particle : Quiz 2}
\author{Sam Meehan}
\date{Due Date : 12 January 2017}

\begin{document}
\maketitle

\textbf{Question 1 : GitHub - You Already did This Once}
Perform the following steps to create GitHub repository.  Feel free to use the internet to search for how to do this if you don't remember.
\begin{itemize}[noitemsep]
\item Create a GitHub repository using the online interface at \href{https://github.com}{github.com} - be sure to include a README.md file when you create this repository.  Call the repository ``quiz1\_NAME'' where NAME is your name.  Remember, \textit{NO SPACES}!
\item Clone this repository to your laptop (\small{\ttfamily{git clone}})
\item Go into this respository directory (\small{\ttfamily{cd}}) and write a file called \textit{mycode.py} that contains some python code.  What that code does is not of importance.
\item Add this file to the repository for it to be tracked (\small{\ttfamily{git add <filename>}}).
\item Check the status of the repository to make sure that the file is ready to be uploaded (\small{\ttfamily{git status}})
\item Commit these changes to the repository (\small{\ttfamily{git commit -m ``Commit message here''}})
\item Push these changes to the external repository on \href{https://github.com}{github.com} (\small{\ttfamily{git push origin master}})
\end{itemize}
When you have pushed this and verified that the file you created exists online, you will submit the URL of this repository when replying to the email.  If you are successful in this, when solving the rest of the questions, please submit the solutions by adding them to your GitHub repository 
\newline
\newline
\textbf{Question 2 : La-La Land Debugging}
This code is meant to print all of the numbers in the list $x$.  Is it correct?  Will it print the five values in $x$ properly?  What will it print? (HINT : It will never print ``Program Finished'')  Use the concepts of representing code with La-La Land to help explain how you understand this code to be working.
\begin{lstlisting}
print ``Program Starting''
x=[3,2,5,6]
x.append(0)
for i in x:
    print i
    x.append(i)
print ``Program Finished''


\end{lstlisting}
You may have to extend the representation in memory so that a single box can contain not just a number but a list of numbers that you keep track of.  You may also need to extend the representation of ``each line'' of code because the line of code within the for loop may execute multiple times.














\end{document}